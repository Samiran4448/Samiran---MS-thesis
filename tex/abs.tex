Subgraph enumeration is an important problem in graph theory with wide range of applications.
Being NP-Complete, this problem needs efficient implementations and smart heuristics to be practical for large sized graphs and queries.
The problem has huge scope for parallelism, there are many existing solutions in the multi-core and distributed computing community.
Graphics Processing Units (GPUs) offer massive parallelism and solutions based on GPUs outperform the multi-core implementations.

Most GPU solutions use Breadth First Traversal to utilize underlying parallelism and impose huge restrictions on hardware due to memory requirements.
PARSEC (Parallel Subgraph Enumeration and Counter) is the first GPU based solution that uses depth first search and performs in memory subgraph enumeration.
In this thesis PARSEC was improved with insights from traditional sequential solutions and smart GPU implementations.

The performance of subgraph Enumeration is limited by number of intersection operations, a smart preprocessing technique was developed that detects scope for intersection reuse and reduces the number of intersections by up to $3.87\times$.
A 2 phases pruning technique was developed which shrinks search space to further reduce the number of intersections by up to $6.6\times$
An in depth analysis of PARSEC was conducted to discover severe load imbalance which limits performance, a hybrid parallelization scheme was developed that improves the load imbalance by a upto $14\times$.
Altogether, these improvements provide a best geometric mean time speedup of $4.6\times$ across data graphs and $3.7\times$ across all queries.
