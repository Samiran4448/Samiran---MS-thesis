\chapter{Conclusions and Future Work}\label{chap:conclusions}

In this thesis, a comprehensive review of subgraph enumeration algorithms using state space search techniques was conducted.
Insights from traditional sequential algorithms and understanding of GPU architecture were used to uncover several enhancements on the state-of-the-art subgraph enumeration application, PARSEC \cite{PARSEC_VD}.
To reduce the number of intersections, the problem of detecting optimal reuse mappings was posed as an optimization problem and a closed form solution was found.
The search space of the search tree was shrunk by using a two-phase hybrid symmetry breaking strategy which uses heuristics to dynamically decide the symmetry breaking direction.
Further, an in-depth analysis of PARSEC \cite{PARSEC_VD} was performed to reveal the underlying load imbalance due to the parallelization scheme and static work division.
The load balance was improved by using a hybrid parallelization scheme that switches after a cutoff.
These different techniques help achieve modest geometric mean speedups of up to $4\times$ across different data graphs and up to $3.7\times$ across all queries. A maximum speedup of $14.6\times$ was attained.
These efforts also help develop numerous computational insights to unlock scope for further improvements.

For future work, the load balance can be improved by using dynamic strategies like work sharing between blocks.
The biggest drawback of this implementation is that it is only applicable for queries with a central node.
Different algorithms need to be developed for supporting more queries.
One idea might be to convert edges in query graphs to vertices to support queries with a central edge.
Another could be to add dummy edges to queries to create an artificial central node.

Another drawback of this implementation is that the memory requirement is in the order of the square of undirected max degree. This restricts the implementation to be truly scalable for large dense graphs. However, real world graphs generally have few vertices with very high degree, these can be dealt with using CPU where the memory is not too restrictive.
Orientation-based techniques were attempted which store induced subgraphs of the oriented graph and find rest of the candidates using binary search.
However, these implementations do not perform well with increasing query size and have even worse load balance.

After addressing the load balance problems, this implementation can be scaled to multiple GPUs by using the CPU unified memory to store the data graph and process even bigger data graphs and larger queries in reasonable time.
Graph partition packages like METIS \cite{metis} can be utilized to scale to multiple CPUs.
This will produce a truly scalable high performing subgraph enumerator unlocking new applications on much larger data graphs.
