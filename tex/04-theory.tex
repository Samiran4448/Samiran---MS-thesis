\chapter{Basic Techniques}\label{chap:basic-theory}

The Subgraph Enumeration algorithm can be decomposed into three major steps: (1) Query Graph Preprocessing, (2) Data Graph Preprocessing, and (3) Search Tree Traversal. This chapter covers each step in detail with a running example.

\begin{figure}
    \includegraphics[width=\textwidth]{fig/LR/sgm-example1.png}
    \caption{Example}
    \label{fig:sgm-example}
\end{figure}
\section{Query Graph Preprocessing}\label{query-preprocessing}
This section explains all processing done on the query graph to enable it for search tree traversal.
The main tasks performed on the input query are: Query sequencing and Symmetry Detection.
Since the query graphs are very small (less than 10 nodes, in general) and all preprocessing tasks are polynomial time complexity, this preprocessing workload is handled by the CPU.
Query preprocessing is important to shrink search tree exploration, the amount of work done at each node, and redundancy elimination.
% \RN{too many details that the readethewould not know about...}
% \samiran{Will try to introduce these in the literature review}.

\subsection{Query Sequencing}
The query graph input to the application is undirected. The first task is to convert this undirected graph to a directed acyclic graph (DAG). The ordering of the DAG governs the order in which these vertices are matched to the data graph.
The query nodes are ordered based on their likelihood of matching with the data graph. Naturally, query nodes with lesser likelihood are prioritized over others.

The query sequencing algorithm used by VF3 \cite{VF3} is utilized for this task. This algorithm uses multiple criteria to estimate the likelihood of matching and sequences the nodes in decreasing order of that likelihood.
Interested readers are encouraged to read about these criteria in \cite{VF3}.
The pseudocode is listed in Algorithm \ref{algo:query-seq}.
The order generated after performing query sequencing on the example query graph is shown in Figure \ref{fig:query-sequencing}.

\begin{algorithm}[h]
    \caption{Query Sequencing}
    \label{algo:query-seq}

    \SetKwData{dm}{$d_M$}
    \SetKwData{deg}{Degree}
    \SetKwData{lh}{likelihood}
    \SetKwData{i}{i}
    \SetKwData{j}{j}
    \SetKwData{idx}{idx}
    \SetKwData{seq}{$S_q$}

    \SetKwFunction{argmax}{argmax}

    \KwIn{Template Graph, $G_q$}
    \KwOut{Node Query Sequence, $S_q$}
    $\dm[N_q] \leftarrow 0$\;
    $S_q \leftarrow \phi$\;
    \For{$\i \leftarrow$ 0 \KwTo $N_q$}{
        \tcp{calculate likelihood for remaining nodes}
        \For{$\j \leftarrow 0$ \KwTo $N_q$}{
            \If{\j not in \seq } {
                $\lh[\j] \leftarrow \dm[\j] \cdot N_q + \deg[\j]$\;
            }
        }
        \tcp{Find node with maximum likelihood and it to \seq}
        \idx $\leftarrow$ \argmax{\lh}\;
        append \idx to \seq\;

        \tcp{Update \dm}
        \ForEach{neighbor \j of node \idx}{$\dm[\j] \leftarrow \dm[\j] + 1$\;}
    }
\end{algorithm}



\begin{figure}
    \includegraphics[width=\textwidth]{fig/LR/query-sequencing.png}
    \caption{Query Sequencing output of $G_q$ using \textit{VF3}}
    \label{fig:query-sequencing}
\end{figure}

\subsection{Symmetry Detection and Breaking}\label{sec:sym-detection}
Symmetry is the existence of automorphism in a graph. Automorphism, as the name suggests, is a graph that is isomorphic to itself.
An automorphism of a graph is a permutation of vertices that maintains edges and non-edges.
% Formally, an automorphism of a graph $G=(V,E)$ is  permutation $\sigma$ of the vertex set $V$ such that the pair of vertices $(u,v)$ form an edge if and only if $(\sigma(u), \sigma(v))$ also forms an edge.
The set of automorphic permutations of the vertex set is called the ``automorphism group".
Figure \ref{fig:automorphism} shows the automorphism group of the query graph $G_q$.
\textit{Orbit} is an equivalence class of vertices in the automorphism group.
\begin{figure}
    \includegraphics[width=\textwidth]{fig/LR/automorphism.png}
    \caption{Automorphism group of $G_q$}
    \label{fig:automorphism}
\end{figure}

The presence of symmetry essentially causes the same subgraph of the data graph to be counted multiple times and hence resulting in extra work.
The easiest technique that can be employed to avoid this is called symmetry breaking. This makes sure that the ties are resolved at the moment they are generated, hence avoiding redundant work.
An easy way to break symmetry is by having an order between the matches of data graph between the symmetric levels, this was first
% \RN{sure? how about previously} %
used by \cite{symbreak}. Algorithm \ref{algo:symbrk} lists the pseudocode for detecting symmetry and generating ordering for symmetry based pruning. Note, the \texttt{GetAutomorphismGroup} and \texttt{GetOrbits} routines used here are provided by NAUTY \cite{nauty}.
% \RN{Where are the functions GetXYZ/}

\begin{algorithm}
    \caption{Symmetry breaking}
    \label{algo:symbrk}
    \SetKwFunction{gag}{GetAutomorphismGroup}
    \SetKwFunction{gorb}{GetOrbits}
    \KwIn{Template Graph, $G_q$}
    \KwOut{Set of partial node orderings, $\mathcal{P}_q$}
    $\mathcal{A} \leftarrow$ \gag{$G_q$}\;
    \While{$\mathcal{A} \neq \mathbb{I}$}{
        $\mathcal{O}_\mathcal{A} \leftarrow$ \gorb{$\mathcal{A}$}\;
        Pick largest orbit $\{v_1, v_2, \dots, v_k\}$ from $\mathcal{O}_\mathcal{A}$\;
        Add $v_1 < v_2, v_1 < v_3, \dots, v_1 < v_k$ to $\mathcal{P}_q$\;
        $\mathcal{A} \leftarrow \{\alpha | \alpha\in\mathcal{A}, \alpha(v_1) = v_1 \}$\;
    }
\end{algorithm}


For example, to avoid redundancy due to automorphism $\{q_1, q_4, q_3, q_2, q_5\}$ of $G_q$, it can be made sure that the data graph vertices matched at level 2 have labels less than the vertices matched at level 4. This is an example of \textit{lexicographic symmetry} breaking.
Another symmetry breaking order can be based on the degree of data graph nodes.
A key observation is that the symmetry breaking criteria can be different for different levels but not always different for different subtrees.
This will be formally established and utilized in Section \ref{sec:hy-symbreak}.

% \iffalse
%     \subsection{Reuse Detection}\label{sec:reuse-detection}
%     Set intersection operation for generating candidates at the next level is the most time-consuming operation in subgraph enumeration, this is a well-known fact in the literature and also re-verified in section \samiran{cite stacked bar graph figure}.
%     While traversing the search tree each node needs a set intersection operation on adjacency lists of backward neighbors (vertices matched with levels connected at the next level).
%     The number of backward neighbors at each level increases with increasing template size. This results in even more adjacency list intersections at each level.
%     RPS \cite{RPS-paper} reduces these operations by generating an intersection reuse plan, this plan smartly finds the intersections that will be required by more than one node and stores them when first calculated. There is a lot of intersection reuse possible with RPS as they employ a BFS strategy.
%     Similar reuse is not possible in DFS traversal since the past information is lost while backtracking.
%     However, Some levels of the sequenced query graphs have similar backward neighbor lists. For such queries' the majority of intersections can be reused by simply storing the intermediate intersection results. This technique is discussed in detail in Section \samiran{cite improvements section}.
%     To do this efficiently, the problem of finding level mappings with the most similar adjacency lists is posed as a linear programming optimization problem.

%     Let, the sequenced query graph be $G_q=(V_q, E_q)$ with $|V_q|=k$ and the vertex sequence $S_q$.
%     $\mathcal{N}(.)$ be the function for getting the backward neighbor list of a vertex.
%     For each pair of vertices at level $i$ and $j$ let $W_{ij}$ be a measure of the commonality between their backward neighbors' lists. Let $X_{i,j}$ be the decision variable which tells if vertex $i$ should reuse intersections from vertex $j$.

%     With these definitions the optimization problem can be modelled as:
%     \begin{align}
%         \max \sum_{i=j+1}^{k}\sum_{j=1}^{k} W_{ij} X_{ij} \\
%         \text{s.t.}
%         \sum_{j=1}^k X_{ij} \leq 1. \quad \forall i \in \{1, \dots, k\}
%     \end{align}
%     Where, $$
%         W_{ij} = \begin{cases}
%             |\mathcal{N}(S_q[i]) \cap \mathcal{N}(S_q[j])| \qquad \text{if} \quad i>j, \mathcal{N}(S_q[i]) \supseteq \mathcal{N}(S_q[j]) \\
%             0   \qquad \text{Otherwise}
%         \end{cases}
%     $$
%     This problem is a linear semi-assignment problem (LSAP) with the optimal solution being the greedy solution itself.
%     This is easy to establish as any other solution can be improved by switching to the greedy solution.
%     The solution to this problem is:
%     $$
%         X_{ij}=\begin{cases}
%             1   \qquad \text{if } j=\argmax_j(w_{ij}>0); \\
%             0   \qquad \text{Otherwise}
%         \end{cases}
%     $$

%     Reuse detection involves a find minimum operation for each level in $G_q$ hence it is polynomial time and can be performed on CPU for small-sized query graphs. Once, this is performed, it gives the levels that are \textit{reusable}. The intersection results for these levels are to be stored, this is discussed in detail in Section \ref{sec:reuse-impl}.

%     \newpage
% \fi

\section{Data Graph Preprocessing}\label{graph-preprocessing}

Data graph preprocessing involves operations on the data graph to shrink the search tree during traversal.
With GPU implementations, preprocessing also helps to improve coalesced memory accesses and memory utilization.


\subsection{Peeling}\label{peeling}
Subgraph Enumeration produces isomorphic matches of query graph in the data graph.
Naturally, each match needs to have degree greater than the degree of the corresponding query node.
This implies that: \textit{Vertices in data graph with degree less than minimum query degree will never form any matches.}
Using this simple observation, all the vertices with degree lesser than the minimum query degree can be deleted from the data graph.
This can also be done iteratively as the lemma holds for the resulting data graph.
This process is commonly called \textit{Peeling}.
Empirical experiments performed by \cite{PARSEC_VD} show that \textit{Peeling} should be performed till at most $5\%$ of nodes are deleted.
Algorithm \ref{algo:peel} shows the pseudocode for the peeling routine.
Note the filtering of vertices is done on GPU using the CUB library \cite{cub}.
\begin{algorithm}[h]
    \caption{Peeling data graph}
    \label{algo:peel}

    \SetKwData{nfs}{PrunedVertices}
    \SetKwData{efs}{PrunedEdges}
    \SetKwFunction{d}{Degree}

    \KwIn{ Minimum query graph degree, $d_{q,min}$, $G_d = (V_d, E_d)$:}
    \Repeat{$|\nfs| < 0.05 \times |V_d|$}{
        \nfs $\leftarrow \{u | u \in V_d, \d{u} < d_{q,min}\}$\;
        \If{$|\nfs| = 0$}{\text{Break;}}
        $\efs \leftarrow \{(u, v) | u \textbf{ or } \in \nfs  $\;
        $V_d \leftarrow V_d \setminus \nfs$\;
        $E_d \leftarrow E_d \setminus \efs$\;
    }
\end{algorithm}
\begin{figure}
    \includegraphics[width=\textwidth]{fig/LR/peeling.png}
    \caption{Peeling Example}
    \label{fig:peeling}
\end{figure}

\subsection{Priority Sorting}\label{sec:prio-sorting}
CSRCOO storage format allows efficient storage of graphs in CPU memory.
As mentioned in Section \ref{storage-format}, the \textit{Column Index} array is sorted by vertex labels to enable binary search.
However, efficient symmetry breaking demands the neighbors to be listed in a different order.
To achieve this, we need another copy of the \textit{Column Index} array in a \textit{priority} sorted fashion while maintaining the same \textit{Row Pointer} boundaries. This array is named \textit{Priority Sorted Column Index} or \textit{PSCI}.
The \textit{priority} can be any property of the vertices; in our case we explore two properties: Degree and Degeneracy.
\begin{figure}
    \includegraphics[width=\textwidth]{fig/LR/prio-sorting.png}
    \caption{Generating Priority Sorted Column Index Array \samiran{Is this image alright?}}
    \label{fig:prio-sorting}
\end{figure}
Given the size of data graph we use GPUs for this preprocessing task.
The \textit{PSCI} array can be generated using a series of two stable sorts, since sorting is fast on GPUs this can be achieved by using the CUB library.
To start with, the \textit{PSCI} array is taken as a copy of the original \textit{Column Index} array.
An array of triplets is then created with each entry containing an element from \textit{PSCI}, \textit{Row Index}, and \textit{priority}. Where, \textit{priority} is an array with entries corresponding to the required property in \textit{PSCI}.
This array of triplets is subjected to two key based stable sorts, the first sort is with the keys being entries in the \textit{priority} array while second sort is with the keys being entries in the \textit{Row Index} array. Figure \ref{fig:prio-sorting} shows the priority sorting performed on $G_d$ as a series of key sorts.

\subsection{Induced Subgraph}\label{encoding}
The search space in subgraph enumeration problem is essentially the whole graph for a general query.
This makes the problem extremely challenging to scale.
However, most practical applications of subgraph isomorphism focus on dense templates.
Similar to \cite{mohammad_K-clique}, we note that restricting the search space can provide immense performance improvements, especially on the GPUs due to enhanced memory utilization.

\begin{figure}
    \includegraphics[width=\textwidth]{fig/LR/Induced-subgraph.png}
    \caption{Induced Subgraph for vertex $d_{12}$}
    \label{fig:induced-subgraph}
\end{figure}

The search space can be restricted by focusing on query graphs containing at least one \textit{Central node}.
A central node is defined as a node in the template graph that is connected to all other nodes.
Under the assumption of a central node, the search space for each match is limited to the subgraph induced by the graph vertex matched to the central node.
Figure \ref{fig:induced-subgraph} shows the subgraph induced by vertex $d_{12}$.
To save storage space, the induced subgraph is stored in a binary adjacency matrix format.
The binary format allows fast set intersection operations as well as efficient symmetry breaking \cite{mohammad_K-clique}.
To use priority based symmetry breaking, the columns of the adjacency matrix are ordered with \textit{Priority Sorted Column Index} Array.

\section{Search Tree Traversal}\label{DFS-T}
% The algorithm was first developed by \cite{ullman_sgm} in $1976$.
% This was a DFS based brute force search technique that uses an adjacency matrix of size $|V_q| \times |V_d|$ as a bijection between an element in the query template and data graph.
% These matrices are recursively generated in a DFS based enumeration process to ultimately generate all the instances.
% The new candidates are generated using the conditions imposed by query graph itself.
% This algorithm assumes a directed query graph and template graph as input.
This section explains how the sequenced query graph $G_q$ and processed data graph $G_d$ are used to perform the search tree traversal for enumerating all matches.
% Let, the directed acyclic query graph be ${G}_q = (V_q, E_q)$ and the processed data graph be $G_d=(V_d, E_d)$.
% We will assume this transformation for the example given in Figure \ref{fig:sgm-example}, the post process query graph is shown in \ref{fig:query-sequencing} and data graph is shown in Figure \ref{fig:peeling}.\\

\begin{algorithm}
    \caption{DFS Traversal}
    \label{algo:DFS-traversal}
    \small
    \SetKwData{l}{$l_{init}$}
    \SetKwData{isubgraph}{$G^d_{ind}$}
    \SetKwData{imatches}{$\texttt{matches}_{init}$}
    \SetKwData{isize}{$\texttt{num\_matches}_{init}$}
    \SetKwData{currIdx}{$\texttt{curr\_idx}$}
    \SetKwData{matches}{$\texttt{matches}$}
    \SetKwData{nmatches}{$\texttt{num\_matches}$}

    \SetKwFunction{intersect}{GenNextCandidates}

    \SetKwData{omatches}{$\texttt{final\_enum}$}
    \SetKwData{ocount}{$\texttt{count}$}

    \KwIn{Sequenced Query Graph: $G_q=(V_q,E_q), S_q$ \newline
        Induced Subgraph: \isubgraph. Initial level: \l \newline
        Initial set of Nodes at \l: \imatches and size \isize
    }
    \KwOut{Set of enumerated matches and count: \omatches, \ocount}
    $l \leftarrow $ \l, $\matches[l] \leftarrow $\imatches, $\nmatches[l] \leftarrow $\isize\;
    $ \currIdx \leftarrow 0$, $\omatches \leftarrow \phi $, $\ocount \leftarrow 0$  \;
    \While{$\currIdx[l] \leq \nmatches[l] $}{

        % $\matches[l+1] \leftarrow \matches[l][\currIdx[l]]$\;
        \intersect()\;
        $\nmatches[l+1]=|\matches[l+1]|$\;
        \If{$(\nmatches > 0 \textbf{ and } l < |V_q|-1$)}{
            $l\leftarrow l+1$\;
            $\currIdx[l]\leftarrow 0$\;
        }
        \ElseIf{$l==|V_q|-1$}{
            \For{$n=\l \textbf{ to } \nmatches[l+1]$}{
                $\texttt{match}\leftarrow \phi$\;
                \For{$k=\l \textbf{ to } l$}{
                    $\texttt{match} \leftarrow \texttt{match} \cup \matches[k][\currIdx[k]]$
                }
                $\omatches \leftarrow \omatches \cup \texttt{match}$
            }
            $\ocount\leftarrow \ocount + \nmatches[l+1]$
        }
        \Else{
            $\currIdx[l]\leftarrow \currIdx[l]+1$\;
            \While{$\currIdx[l]==\nmatches[l] \textbf{ and } l > \l$ }{
                $l\leftarrow l-1$\;
                $currIdx[l]\leftarrow \currIdx[l] +1 $\;
            }
        }

    }
\end{algorithm}

Algorithm \ref{algo:DFS-traversal} describes the DFS based iterative search tree traversal process.
The whole traversal can work within a DFS stack of size $|V_q|\times Degree_{max}(G_d)$.
At each node of the search tree, \textit{GetNextNodes} function is called which generates the candidates to be visited in the next level.
This function is described in Algorithm \ref{algo:intersect}.
It performs an adjacency list intersection operation to generate all possible candidates for the next level (lines 2 - 5).
Since, there might be redundant candidates generated due to the symmetry of the query, some of them need to be pruned. This is done by the next loop (lines 6 - 8).
Note that the set $\mathcal{V}_{orient}$ is available without overheads due to column order of the induced subgraph.

\begin{algorithm}
    \caption{Generate Candidates for Next Level}
    \label{algo:intersect}
    \SetKwData{isubgraph}{$G^d_{ind}$}
    \SetKwData{currIdx}{$\texttt{curr\_idx}$}
    \SetKwData{matches}{$\texttt{matches}$}
    \SetKwData{nmatches}{$\texttt{num\_matches}$}
    \SetKwFunction{intersect}{GenNextCandidates}
    \KwIn{Induced Subgraph \isubgraph, level $l$ \newline
        Backward Neighbor List $\mathcal{N'}$ \newline
        Symmetry levels List $\mathcal{S}$ \newline
        Set of vertices of lower priority than given vertex $\mathcal{V}_{orient}()$ \newline
        List of matches in previous levels $\matches[]$
    }
    $\texttt{Function }\intersect() $\newline
    \While{$(\currIdx[l]<\nmatches[l])$}{
    \ForAll{$k \in \mathcal{N'}(S_q[l+1])$}{
        $u \leftarrow \matches[k][\currIdx[k]]$\;
        $\matches[l+1] \leftarrow \matches[l+1]\cap \mathcal{N}(\isubgraph(u))$\;
        }
        \ForAll{$j \in \mathcal{S}(S_q[l+1])$}{
            $u \leftarrow \matches[j]$\;
            $\matches[l+1]\leftarrow \matches[l+1] \setminus \mathcal{V}_{orient}(u)$\;
        }
    }
    $\texttt{Function End}$
\end{algorithm}

We now walk through the Search tree traversal algorithm with the example query graph sequence given in Figure \ref{fig:query-sequencing} and the peeled data graph in Figure \ref{fig:peeling}.
% Note, the pseudocode gives a DFS-based traversal tree but since it does not store the whole tree, we resort to a BFS traversal here for enumerating the whole exploration tree. \RN{language!}
Note, the pseudocode in Algorithm \ref{algo:DFS-traversal} describes a DFS-based traversal technique while the example enumerates search tree traversal in a BFS manner for illustration purposes.

\begin{enumerate}[Step 1:]
    \item Select candidates in $G_d$ with degree greater than or equal to $q_1$. Here only vertices $\{d_1, d_8, d_9, d_{12}\}$ will be selected. All these are matched to $q_1$ and traversed separately. For conciseness, we will only consider vertices $d_1$ and $d_{12}$. These matches are shown in Figure \ref{fig:sgm-step2}.
          \begin{figure}
              \includegraphics[width=\textwidth]{fig/LR/sgm-step2.png}
              \caption{Step 1}
              \label{fig:sgm-step2}
          \end{figure}
    \item For candidates of level 2, select all neighbors of vertex matched at level 1 with degree greater than that of $q_2$, i.e., 3. As shown in Figure\ \ref{fig:sgm-step3} for the subtree rooted at $d_1$, there will be 5 candidates: $\{d_2, d_7, d_8, d_9, d_{10}, d_{12}\}$. Out of these 5 candidates, $d_7$ is pruned since, it has degree less than $q_2$.\\
          \begin{figure}
              \includegraphics[width=\textwidth]{fig/LR/sgm-step3.png}
              \caption{Step 2}
              \label{fig:sgm-step3}
          \end{figure}
          For subtree rooted at $d_{12}$, there will be 4 candidates: $\{d_1, d_2, d_8, d_{10}\}$ and none can be pruned.
    \item Level 3 candidates are obtained by the intersection of matches at level 2 and level 1. Since this level is symmetric to level 2 symmetry breaking needs to be performed. We use the decreasing degree criteria for symmetry breaking in this example.
          \begin{figure}
              \includegraphics[width=\textwidth]{fig/LR/sgm-step4.png}
              \caption{Step 3}
              \label{fig:sgm-step4}
          \end{figure}
          Figure \ref{fig:sgm-step4} shows all candidates at level 3. The candidates marked in red are pruned.
          For example 9 and 12 is pruned in subtree ${d_1, d_8, \dots}$ since $Degree(d_9)<Degree(d_8)$.
    \item For level 4 candidates, the adjacency intersection of $q_1$ and $q_3$ is required. This level is also symmetric to level 2. The subtree rooted at $d_{12}$ does not yield any candidates. While the other gets one candidate each. Symmetry Breaking does not prune any candidates here.
          \begin{figure}
              \includegraphics[width=\textwidth]{fig/LR/sgm-step5.png}
              \caption{Step 4}
              \label{fig:sgm-step5}
          \end{figure}
    \item Level 5 candidates (or final matches) are generated by the adjacency intersection $q_1$, $q_2$, and $q_4$. This level is symmetric to both $q_2$ and $q_3$.
          The candidates generated by intersection at the first leg are $\{d_8, d_{10}\}$, symmetry breaking will eliminate both the candidates as $Degree(d_2)<Degree(d_8)$. \RN{DONE TILL HERE.} There is a tie between $d_2$ and $d_{10}$ which is resolved \textit{lexicographically}.
          \begin{figure}
              \includegraphics[width=\textwidth]{fig/LR/sgm-step6.png}
              \caption{Step 5}
              \label{fig:sgm-step6}
          \end{figure}
          Similarly, the other leg generates 2 valid candidates for level 5.
\end{enumerate}
This way the search tree traversal process terminates with final matches
$\{d_1, d_9, d_8, d_{12}, d_2\}$ and $\{d_1, d_9, d_8, d_{12}, d_{10}\}$.
There might be more matches originating from subtrees rooted at $\{d_8, d_9\}$ which were not illustrated here.

To conclude, this chapter gives a detailed understanding of all techniques involved in subgraph enumeration. As we can see, all subtrees are independent of each other hence, they can be easily parallelized on GPUs.

One more subtle observation to be made here is that the exploration is dependent on the symmetry breaking technique.
We can see from Figure \ref{fig:sgm-step4} that there would have been many more candidates at level 4 if the symmetry breaking was done using lexicographic criteria or increasing degree criteria.
These candidates would have ultimately pruned in lower levels, resulting in extra work.
Therefore, selecting an efficient symmetry-breaking strategy is important and discussed in detail later in Section \ref{sec:hy-symbreak}.
